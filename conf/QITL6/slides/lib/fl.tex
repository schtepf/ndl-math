%%
%% some useful macros for formal languages
%%

%% empty word \eps (length \abs{w} defined in math.tex)
\newcommand{\eps}{\epsilon}

%% roman (or sans-serif) letters used as symbols in examples
\newcommand{\z}[1]{\textsf{#1}}

%% language denoted by a given grammar or regular expression
\newcommand{\lang}[2][]{\mathcal{L}_{#1}\bigl[#2\bigr]}

%% language generated by a Turing machine 
\newcommand{\generate}[2][]{\mathcal{G}_{#1}\bigl[#2\bigr]}

%% regular expressions: \reg{expr}, \R(, \R), \R*, ..., \len{\reg{expr}}
\newcommand{\R}[1]{\texttt{#1}}
\newcommand{\reg}[1]{\underline{#1}}
\newcommand{\regeps}{\reg{\eps}}
%\newcommand{\len}[1]{\ell\left(#1\right)} %% do we still need that command?

%% class \langReg[\Sigma] of all regular languages, and set \langR[\Sigma] of all regular expressions
\newcommand{\langReg}[1][\Sigma]{\operatorname{Reg}(#1)}
\newcommand{\langR}[1][\Sigma]{R(#1)}


%%
%% Endliche Automaten
%%


%% Regeln f�r die formale Beschreibung (im Formelmodus zu verwenden):
%%   \Xearule{<lab1>}{<lab2>}{<label>}
%% [<lab1>, <lab2> sind Zustandslabel und werden im Formelmodus gesetzt]
%% [<label> bezeichnet das Eingabezeichen und wird AUCH im FORMELMODUS gesetzt]
%%   \earule{<n1>}{<n2>}{<label>}
%% [<label> bezeichnet das Label einer Bewegung und wird im Textmodus gesetzt]
\newcommand{\earule}[3]{q_{#1} \overset{\text{#3}}{\longrightarrow} q_{#2}}
\newcommand{\Xearule}[3]{#1 \overset{#3}{\longrightarrow} #2}



%%
%% Kontextfreie Grammatiken
%%

%% Ableitungsschritte und Ableitung
%%   \drv[<grammar>]
%%   \Drv[<grammar>]
\newcommand{\drv}[1][]{\implies_{#1}}
\newcommand{\Drv}[1][]{\implies_{#1}^*}

%% class \langKF[\Sigma] of all context-free languages
\newcommand{\langKF}[1][\Sigma]{\operatorname{KF}(#1)}

%% Regeln f"ur die formale Beschreibung von Kellerautomaten (im Formelmodus):
%%   \Xpdarule{<state1>}{<V>}{<label>}{<state2>}{<stack>}
%% [�bergang von Zustand <state1> mit oberem Kellersymbol <V> in
%%  Zustand <state2>, wobei <stack> auf dem Keller abgelegt wird;
%%  dabei wird das Zeichen <label> verarbeitet]
\newcommand{\Xpdarule}[5]{(#1,#2) \overset{#3}{\longrightarrow} (#4,#5)}

%% A \to \alpha \oder \beta ... Disjunktionsstrich mit zus�tzlichem Abstand
\newcommand{\oder}{\,|\,}



%%% Local Variables: 
%%% mode: latex
%%% TeX-master: ""
%%% End: 
