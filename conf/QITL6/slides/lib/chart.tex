%%
%% macros for demonstrating chart parsers with PGF
%%

\usepackage{calc,ifthen}

%% configure chart dimensions:
%%   \setlength{\chartXunit}{5mm}, \setlength{\chartYunit}{5mm}
%%   \setChartWidth{8}
\newlength{\chartXunit}
\setlength{\chartXunit}{10mm}
\newlength{\chartYunit}
\setlength{\chartYunit}{5mm}
\newcounter{ChartWidth}
\setcounter{ChartWidth}{8}
\newcommand{\setChartWidth}[1]{%
  \setcounter{ChartWidth}{#1}}

%% these are temporary registers for internal use
\newlength{\chartTemp}
\newlength{\chartTempH}
\newlength{\chartTempW}
\newlength{\chartTempX}
\newlength{\chartTempY}

%% internal: \chartXcoord{<x>} -> \chartTempX
\newcommand{\chartXcoord}[1]{%
  \setlength{\chartTempX}{\chartXunit * \real{#1}}%
}

%% internal: \chartYcoord{<y>} -> \chartTempY
\newcommand{\chartYcoord}[1]{%
  \setlength{\chartTempY}{\chartYunit * \real{#1}}%
}

%% internal: \chartXYcoord{<x>}{<y>} -> \chartTempX, \chartTempY
\newcommand{\chartXYcoord}[2]{%
  \chartXcoord{#1}\chartYcoord{#2}%
}

%% internal: insert strut for full text line height (to improve text alignment)
\newcommand{\chartStrut}{%
  \settoheight{\chartTemp}{X}\rule{0mm}{\chartTemp}%
}
 
%% put terminal symbol <Text> in given <Slot> (x-coordinate) with optional label
%%   \chartTerminal<overlay>[Lab]{Slot}{Text}
\newcommand<>{\chartTerminal}[3][chartNULL]{%
  \only#4{%
    \chartXYcoord{#2}{0}
    \settowidth{\chartTempW}{#3}
    \pgfputat{\pgfpoint{\chartTempX}{\chartTempY}}{\pgfbox[center,base]{\chartStrut{}#3}}%
    \addtolength{\chartTempY}{.4em}%
    \pgfnoderect{#1}[virtual]{\pgfpoint{\chartTempX}{\chartTempY}}{%
      \pgfpoint{\chartXunit / 2}{1em}}%
  }
}

%% draw arrow across span of substring at height <Y> and label with <Tag>
%%   \chartSpan[Lab]{Y}{X1}{X2}{Tag}
\newcommand<>{\chartSpan}[5][chartNULL]{%
  \only#6{%
    \chartXYcoord{#3}{#2}%
    \setlength{\chartTemp}{\chartTempX}%
    \chartXcoord{#4}%
    \addtolength{\chartTemp}{\chartXunit * \real{-.35}}%
    \addtolength{\chartTempX}{\chartXunit * \real{.35}}%
    \begin{pgfscope}
      \pgfclearstartarrow{}\pgfsetendarrow{\pgfarrowtriangle{3pt}}%
      \pgfsetlinewidth{1pt}%
      \color{counterpoint}%
      \pgfline{\pgfpoint{\chartTemp}{\chartTempY}}{\pgfpoint{\chartTempX}{\chartTempY}}%
    \end{pgfscope}
    \setlength{\chartTempH}{\chartTempY + \chartYunit * \real{0.4}}%
    \setlength{\chartTempW}{\chartTempX - \chartTemp}%
    \setlength{\chartTempX}{(\chartTemp + \chartTempX) / 2}%
    \pgfnoderect{#1}[virtual]{\pgfpoint{\chartTempX}{\chartTempH}}{\pgfpoint{\chartTempW}{\chartYunit * \real{0.8}}}%
    \addtolength{\chartTempY}{2pt}%
    \pgfputat{\pgfpoint{\chartTempX}{\chartTempY}}{%
      \pgfbox[center,base]{\scriptsize\color{secondary}#5}}%
  }%  
}

%% CFG production for use as span label (LHS in bold)
%%   \charProd{LHS}{RHS}
\newcommand{\chartProd}[2]{%
  \textbf{#1} $\to$ {#2}%
}

%% draw dashed line across entire chart at height <Y>, optional with <Text> on LHS
%%   \chartRule{Y}, \chartRule[Text]{Y}
\newcommand<>{\chartRule}[2][]{%
  \only#3{%
    \begin{pgfscope}
      \chartXYcoord{0}{#2}%
      \setlength{\chartTemp}{\chartTempX}%
      \chartXcoord{\theChartWidth}%
      \addtolength{\chartTemp}{\chartXunit * \real{-.5}}%
      \addtolength{\chartTempX}{\chartXunit * \real{-.5}}%
      \addtolength{\chartTempY}{\chartYunit * \real{.4}}%
      \pgfclearstartarrow{}\pgfclearendarrow{}%
      \pgfsetlinewidth{1pt}%
      \pgfsetdash{{2mm}{2mm}}{1mm}%
      \pgfline{\pgfpoint{\chartTemp}{\chartTempY}}{\pgfpoint{\chartTempX}{\chartTempY}}%
      \addtolength{\chartTemp}{-4pt}%
      \pgfputat{\pgfpoint{\chartTemp}{\chartTempY}}{\pgfbox[right,base]{\small #1}}%
    \end{pgfscope}
  }%  
}

%% draw edge of parse tree into chart (nodes must have been labelled);
%% uses current color (change with \color{...} before drawing edges)
%%   \chartEdge<overlay>{Lab1}{Lab2}
\newcommand<>{\chartEdge}[2]{%
  \begin{pgfscope}
    \pgfclearstartarrow%
    \pgfclearendarrow%
    \pgfsetlinewidth{1.5pt}%
    \pgfnodesetsepstart{2pt}%
    \pgfnodeconnline#3{#1}{#2}%
  \end{pgfscope}
}

%%% Local Variables: 
%%% mode: latex
%%% TeX-master: ""
%%% End: 
