\begin{frame}
  \frametitle{Objectives}

  \begin{itemize}
  \item Explain the mathematical foundations of Naïve Discriminative Learning (NDL) in one place and in a consistent way
  \item Highlight the theoretical similarities of NDL with linear/logistic regression and the single-layer perceptron
  \item Present some empirical simulations of stochastic NDL learners, in light of the theoretical insights
  \end{itemize}
\end{frame}

\begin{frame}
  \frametitle{Naïve Discriminative Learning}
  
  \begin{itemize}
  \item \citet{Baayen:11,Baayen:etc:11}
  \item Incremental learning equations for direct associations between cues and outcomes \citep{Rescorla:Wagner:72} 
  \item Equilibrium conditions \citep{Danks:03}
  \item Implementation as R package \texttt{ndl} \citep{Arppe:etc:14}
  \end{itemize}
  
  \gap[1]
  \begin{description}[Discriminative:]
   \item[Naive:] cue-outcome associations estimated separately for
    each outcome (this independence assumption is similar to
    a naive Bayesian classifier)
  \item[Discriminative:] cues predict outcomes based on total activation level
    = sum of direct cue-outcome associations
  \item[Learning:] incremental learning of association strengths
  \end{description}
\end{frame}

\begin{frame}
  \frametitle{The Rescorla-Wagner equations (1972)}

  Represent incremental associative learning and subsequent on-going
  adjustments to an accumulating body of knowledge.

  \gap[1]
  Changes in cue-outcome association strengths:
  \begin{itemize}
  \item No change if a cue is not present in the input
  \item Increased if the cue and outcome co-occur
  \item Decreased if the cue occurs without the outcome
  \item If outcome can already be predicted well (based on all input cues),
    adjustments become smaller
  \end{itemize}

  \gap[1] 
  Only results of incremental adjustments to the cue-outcome associations are
  kept -- no need for remembering the individual adjustments, however many
  there are.
\end{frame}

\begin{frame}
  \frametitle{Danks (2003) equilibrium conditions} 

  \begin{itemize}
  \item Presume an ideal stable ``adult'' state, where all cue-outcome
    associations have been fully learnt -- further data points should then
    have no impact on the cue-outcome associations
  \item Provide a convenient short-cut to calculating the final cue-outcome
    association weights resulting from incremental learning, using relatively
    simple matrix algebra
  \item Most learning parameters of the Rescorla-Wagner equations drop
    out of the Danks equilibrium equation
  \item Circumvent the problem that a simulation of an R-W learner does
    usually not converge to a stable state unless the learning rate is
    gradually decreased
  \end{itemize}
\end{frame}

\begin{frame}
  \frametitle{Traditional \vs linguistic applications of R-W}

  \begin{itemize}
  \item Traditionally: simple controlled experiments on item-by-item
    learning, with only a handful of cues and perfect associations
  \item Natural language: full of choices among multiple possible alternatives
    -- phones, words, or constructions -- which are influenced by a large
    number of contextual factors, and which often show weak to moderate 
    tendencies towards one or more of the alternatives rather than a
    single unambiguous decision
  \item These messy, complex types of problems are a key area of interest in
    modeling and understanding language use
  \item Application of R-W in the form of a Naïve Discriminative Learner to
    such linguistic classification problems is sucessful in practice and can 
    throw new light on research questions
  \end{itemize}
\end{frame}

\begin{frame}
  \frametitle{Related work}

  \begin{itemize}
  \item R-W \vs perceptron \citep[p.~155f]{Sutton:Barto:81}
  \item R-W \vs least-squares regression \citep[p.~457]{Stone:86}
  \item R-W \vs logistic regression \citep[p.~234]{Gluck:Bower:88}
  \item R-W \vs neural networks \citep{Dawson:08}
  \item[\hand] similarities are also mentioned by many other authors \ldots
  \end{itemize}
\end{frame}

%%% Local Variables: 
%%% mode: latex
%%% TeX-master: "../qitl6_evert_arppe"
%%% End: 
