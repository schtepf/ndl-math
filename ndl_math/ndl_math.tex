%%
%%  Template for a KLUE tech report
%%

\documentclass[a4paper,11pt]{article} % other options: twoside, 10pt, 12pt

%% -- set up page size and margins
\usepackage[a4paper,
  top=30mm, bottom=30mm, left=30mm, right=30mm,
  head=15pt, foot=30pt
]{geometry}

%% -- header and footer
\usepackage{fancyhdr}
\pagestyle{fancy}                    % these header/footer settings are just an example
\lhead{\textsc{KLUE Technical Report}}
\chead{WS 2014/15}
\rhead{PS Computerlinguistik}
\lfoot{}
\cfoot{\thepage}
\rfoot{}
\renewcommand{\headrulewidth}{0.4pt}
\renewcommand{\footrulewidth}{0pt}

%% -- set up language and encoding
\usepackage[english]{babel}          % language settings
\usepackage[utf8]{inputenc}          % so you can write non-ASCII characters in the input text
\usepackage[T1]{fontenc}             % use the modern T1 font encoding

%% -- uncomment these lines if you prefer a small gap between paragraphs
%% \setlength{\parindent}{0mm}
%% \setlength{\parskip}{\medskipamount}

%% -- recommended packages
\usepackage{amsmath,amssymb,amsthm}  % AMS math extensions
\usepackage{graphicx,rotating}       % basic graphics, rotated text 
\usepackage{xcolor}                  % modern colour support
\usepackage{hyperref,url}            % add links to PDFs, \url{} command
\usepackage{array,hhline,booktabs}   % modern tabular displays
\usepackage{enumitem}                % customized lists
\usepackage{microtype}               % optimized typesetting
\usepackage[textwidth=25mm,textsize=footnotesize]{todonotes}               % you'll come to love this

%% -- optional packages
%% \usepackage{pifont}               % Adobe Dingbats with \ding{164}
%% \usepackage{tikz}                 % create sophisticated drawings (once you've read the 1100-page manual)
%% \usepackage[normalem]{ulem}       % for better underlining and crossed-out text
%% \usepackage{multirow}             % cells spanning multiple rows in tabular display
%% \usepackage{alltt}                % verbatim text with latex macros

%% -- set up bibliography with flexible natbib style
\usepackage{natbib}
\bibpunct{(}{)}{;}{a}{}{,}           % configure format of citations
 
%% -- choose a font set (done last to avoid clashes with other packages)
%% the default Computer Modern font is perfectly fine, but it has rather low weight for screen reading
%% \usepackage[varg]{pxfonts}        % Palatino with matching math font
%% \usepackage[varg]{txfonts}        % Adobe Times with matching math font
%% \usepackage{mathptmx}             %   mostly uses standard PDF fonts
%% \usepackage[charter]{mathdesign}  % Charter with matching math font
%% \usepackage{fourier}              % Utopia with matching math font
%% \usepackage{arev}                 % A sans-serif font (Bitstream Vera) with matching maths
%% see http://www.tug.org/pracjourn/2006-1/hartke/ for font samples

%% -- my own macros
\usepackage{marginnote}         % make \todo{}'s work in footnotes
\renewcommand{\marginpar}{\marginnote}

\newcommand{\tocite}[1]{{\todo{citation}\color{red}[#1]}}
\newcommand{\psup}[1]{\ensuremath{^{(#1)}}}
\newcommand{\psupt}{\psup{t}}

%%%%%%%%%%%%%%%%%%%%%%%%%%%%%%%%%%%%%%%%%%%%%%%%%%%%%%%%%%%%%%%%%%%%%%%%
\begin{document}

%% -- typeset title and author, or create a separate titlepage
\title{Some mathematical observations on the Naive Discriminative Learner:
  Rescorla-Wagner vs.\ single-layer perceptron vs.\ least-squares regression}
\author{
  Stefan Evert\\
  FAU Erlangen-Nürnberg
}
\date{17 August 2015}
\maketitle

%% if you want a customized titlepage, typeset everything yourself in a
%% \begin{titlepage} ... \end{titlepage} environment

%% -- generate a table of contents
\tableofcontents{}
%% \newpage                          % start new page after long toc


%%%%%%%%%%%%%%%%%%%%%%%%%%%%%%%%%%%%%%%%%%%%%%%%%%%%%%%%%%%%%%%%%%%%%%%%
%% -- the main body of your report

\section{Introduction}
\label{sec:intro}

Naive Discriminative Learning \citep[NDL,][]{Baayen:11} performs linguistic classification on the basis of direct associations between cues and outcomes, which are learned incrementally according to the Rescorla-Wagner (R-W) equations \citep{Rescorla:Wagner:72}. \citet{Danks:03} argued that if the R-W equations successfully acquire the true associations between cues and outcomes, they should approximate an equilibrium state in which the expected change in association values is zero if a cue-outcome event is randomly presented to the learner.  This equilibrium state can be computed directly by solving a matrix equation, without carrying out many iterations of R-W updates, making NDL attractive as an efficient learning technique for quantitative linguistics. Use of the Danks equilibrium also circumvents the problem that a simulation of the R-W model does not converge to a single state unless the learning rate is gradually reduced.

It is well known that the R-W model is closely related to neural networks (through the ``delta rule'' for gradient-descent training of a single-layer perceptron) and to linear least-squares regression \citep[e.g.][]{Danks:03,Baayen:11}.  However, most authors do not seem to be aware of the true depth of these similarities and of their implications.

In this paper, we show that the R-W equations are identical to gradient-descent training of a single-layer feed-forward neural network, which we refer to as a single layer perceptron (SLP\footnote{The term SLP is often reserved for a particular form of such a single-layer network using a Heavyside activation function (cf.\ \url{https://en.wikipedia.org/wiki/Perceptron}).  Here, we use it more generally to refer to any single-layer feed-forward network.}) here (Sec.~\ref{sec:slp}). Based on this result, we present a new, simpler derivation of the equilibrium conditions \citep{Danks:03} and prove that they correspond to the solution of a linear least-squares regression problem (Sec.~\ref{sec:regression}).  In Sec.~\ref{sec:consequence} we discuss some consequences of these new insights.

\section{The Rescorla-Wagner equations}
\label{sec:math-def}

This section gives a mathematically precise definition of the R-W model, following the notation of \citet{Danks:03}.  The purpose of the R-W equations is to determine associations between a set of cues $C_1, \ldots, C_n$ and a single outcome $O$ in a population of event tokens $(\mathbf{c}\psupt, o\psupt)$, where $\mathbf{c}\psupt = (c_1\psupt, \ldots, c_n\psupt)$ is a vector of cue indicators for event $t$ and $o\psupt$ an indicator for the outcome $O$.  Formally,
\begin{align}
  \label{eq:def-c-o}
  c_i\psupt &= 
  \begin{cases}
    1 & \text{if $C_i$ is present} \\
    0 & \text{otherwise}
  \end{cases}
  &
  o\psupt &= 
  \begin{cases}
    1 & \text{if $O$ results} \\
    0 & \text{otherwise}
  \end{cases}
\end{align}
for $t = 1, \ldots, m$ (and $m = \infty$ can be allowed).

When presented with an event $(\mathbf{c}, o)$, the R-W equations update the associations $V_i$ between cues and the outcome according to Eq.~(\ref{eq:def-rw}), which is a more formal notation of Eq.~(1) from \citet[111]{Danks:03}.
\begin{equation}
  \label{eq:def-rw}
  \Delta V_i =
  \begin{cases}
    0 & \text{if } c_i = 0\\
    \alpha_i \beta_1 \bigl(\lambda - \sum_{j=1}^n c_j V_j \bigr) & \text{if } c_i = 1 \wedge o = 1 \\
    \alpha_i \beta_2 \bigl(0 - \sum_{j=1}^n c_j V_j \bigr) & \text{if } c_i = 1 \wedge o = 0
  \end{cases}
\end{equation}
Here, $\alpha_i$ is a measure of the salience of cue $C_i$, $\beta_1$ and $\beta_2$ are learning rates for positive ($o=1$) and negative ($o=0$) events, and $\lambda$ is the maximal activation level of the outcome $O$.  A simplified form of the R-W equations proposed by \citep{Widrow:Hoff:60} assumes that $a_1 = \dots = a_n = 1$,  $\beta_1 = \beta_2 = \beta$ and $\lambda = 1$ (known as the W-H rule).

\citet{Danks:03} argues that a successful R-W model should approach an equilibrium state of the association vector $\mathbf{V} = (V_1, \ldots, V_n)$ where the expected update $E[\Delta V_i] = 0$ if a random event token is sampled from the population.  If we make the simplifying assumption that $\beta_1 = \beta_2 = \beta$, this condition corresponds to the equality
\begin{equation}
  \label{eq:danks-equi}
  \lambda \frac1m \sum_{t=1}^m c_i\psupt o_i\psupt - \sum_{j=1}^n V_j \frac1m \sum_{t=1}^m c_i\psupt c_j\psupt = 0
\end{equation}
In Danks's notation, Eq.~(\ref{eq:danks-equi}) can be written as
\begin{equation}
  \label{eq:danks-probs}
  \lambda P(O, C_i) - \sum_{j=1}^n V_j P(C_i, C_j) = 0
\end{equation}
and is equal to his Eq.~(11) \citep[113]{Danks:03} multiplied by $P(C_i)$.

\section{R-W and the single layer perceptron}
\label{sec:slp}

We will now formulate a single-layer feed-forward neural network (SLP) whose learning behaviour -- with gradient-descent training, which corresponds to the backprop algorithm for a SLP and is also known as the ``delta rule'' in this case -- is identical to the R-W equations with equal positive and negative learning rates $\beta_1 = \beta_2$ (but no other restrictions).  The SLP requires a slightly different representation of events as pairs $(\mathbf{x}\psupt, z\psupt)$ with
\begin{align}
  \label{eq:def-x-z}
  x_i\psupt &= 
  \begin{cases}
    a_i & \text{if $C_i$ is present} \\
    0 & \text{otherwise}
  \end{cases}
  &
  z\psupt &= 
  \begin{cases}
    \lambda & \text{if $O$ results} \\
    0 & \text{otherwise}
  \end{cases}
\end{align} 
Here, $a_i > 0$ is a (different) measure of the salience of cue $C_i$ and $\lambda > 0$ the maximum activation of outcome $O$.  Note that the event representation $(\mathbf{x}, z)$ is connected to the representation $(\mathbf{c}, o)$ through the equivalences $x_i = a_i c_i$ and $z = \lambda o$.  In the W-H case, the two representations are identical.

The SLP computes the activation of the outcome as a linear combination $y = \sum_{i=1}^n w_i x_i$ of the input variables, where $\mathbf{w} = (w_1, \ldots, w_n)$ is the weight vector of the network.  It uses a linear activation function $h(y) = y$ and a Euclidean cost function for the difference between $y$ and the desired activation level $z$.  The cost associated with a given event token $(\mathbf{x}, z)$ is thus
\begin{equation}
  \label{eq:cost-single}
  E(\mathbf{w}, \mathbf{x}, z) = (z - y)^2 = \biggl( z - \sum_{i=1}^n w_i x_i \biggr)^2 .
\end{equation}
For batch updates based on the full population of event tokens, the corresponding batch cost is
\begin{equation}
  \label{eq:cost-batch}
  E(\mathbf{w}) = \sum_{t=1}^m E(\mathbf{w}, \mathbf{x}\psupt, z\psupt) .
\end{equation}
If smaller batches are used, the sum $j$ ranges over a subset of the population for each update step.

Presented with an event token $(\mathbf{x}, z)$, gradient-descent training of this SLP updates the weight vector by 
\begin{equation}
  \label{eq:gd-rule}
  \Delta w_i = -\delta \frac{\partial E(\mathbf{w}, \mathbf{x}, z)}{\partial w_i}
\end{equation}
where $\delta > 0$ is the learning rate and the gradient $\partial E / \partial w_i$ is given by
\begin{equation}
  \label{eq:gd-gradient}
  \frac{\partial E(\mathbf{w}, \mathbf{x}, z)}{\partial w_i} = 2 (z - y) (-x_i)
  = -2 \biggl( z - \sum_{j=1}^n w_j x_j \biggr) x_i .
\end{equation}
%%
Inserting the equalities $x_i = a_i c_i$ and $z = \lambda o$, we obtain
\begin{equation}
  \label{eq:slp-delta}
    \Delta w_i =
    \begin{cases}
      0 & \text{if } c_i = 0\\
      2 \delta a_i \bigl(\lambda - \sum_{j=1}^n c_j a_j w_j \bigr) & \text{if } c_i = 1 \wedge o = 1 \\
      2 \delta a_i \bigl(0 - \sum_{j=1}^n c_j a_j w_j \bigr) & \text{if } c_i = 1 \wedge o = 0 
    \end{cases}
\end{equation}
Comparing this with Eq.~(\ref{eq:def-rw}), we can set $V_j = a_j w_j$, i.e.\ we interpret the weight vector $\mathbf{w}$ of the SLP as salience-adjusted cue-outcome associations.  With $\Delta V_i = a_i \Delta w_i$, we obtain
\begin{equation}
    \Delta V_i =
    \begin{cases}
      0 & \text{if } c_i = 0\\
      2 \delta a_i^2 \bigl(\lambda - \sum_{j=1}^n c_j V_j \bigr) & \text{if } c_i = 1 \wedge o = 1 \\
      2 \delta a_i^2 \bigl(0 - \sum_{j=1}^n c_j V_j \bigr) & \text{if } c_i = 1 \wedge o = 0
    \end{cases}
\end{equation}
which is identical to the R-W equations for $\beta_1 = \beta_2 = 2\delta$ and $\alpha_i = a_i^2$.

The assumption $\beta_1 = \beta_2$ can be relaxed if we change the representation of events to
\begin{equation}
  \label{eq:def-x-mod}
  x_i\psupt = 
  \begin{cases}
    a_i & \text{if } c_i = 1 \wedge o = 1 \\
    a_i \sqrt{\frac{\beta_2}{\beta_1}} & \text{if } c_i = 1 \wedge o = 0 \\
    0 & \text{otherwise}
  \end{cases}
\end{equation}
i.e.\ if we allow the salience of cues to differ between positive ($o = 1$) and negative ($o = 0$) events; the scaling factor $\beta_2 / \beta_1$ is the same for all cues $C_i$.  We do not pursue this extension further here because it affects the equilibrium state in an unpredictable way.  As \citet{Danks:03} has already observed, the cue saliences $\alpha_i$ have no impact at all on the equilibrium and the maximum activation level $\lambda$ merely results in a linear scaling. 

\section{R-W and least-squares regression}
\label{sec:regression}

We have shown in Sec.~\ref{sec:slp} that the R-W equations describe the gradient-descent training of a SLP for the linear regression problem
\begin{equation}
  \label{eq:lsr-def}
  \min_{\mathbf{w}} E(\mathbf{w})
  = \min_{\mathbf{w}} \sum_{t=1}^m E(\mathbf{w}, \mathbf{x}\psupt, z\psupt) .
\end{equation}
This equivalence holds generally, not only in the case of the simplified W-H rule.  Thus, both R-W and our SLP aim to solve the same regression.

If the training procedure is successful, the weight vector $\mathbf{w}$ should approach the least-squares solution of the regression problem.  With single updates (corresponding to the R-W model), convergence cannot be achieved unless the learning rate is gradually reduced.  With batch updates treating the entire population as a single batch, the cost $E(\mathbf{w})$ is a convex function of $\mathbf{w}$ and the gradient descent procedure converges to its unique minimum after a sufficient number of iterations.\footnote{In fact, the minimum of $E(\mathbf{w})$ might not be unique under certain circumstances, viz. if the correlation matrix of the cues is not positive definite; cf. \citet[115--116]{danks2003} for the special case of ``coextensive'' cues. In order to keep the discussion straightforward, we assume the general case of a unique minimum in the present paper.}

In order to express Eq.~(\ref{eq:lsr-def}) more concisely, we define an $m\times n$ matrix $\mathbf{X} = (x_i\psupt) = (x_{ti})$ of cues for all event tokens in the population.  The rows of this matrix correspond to event tokens $t$, the columns to cues $i$; i.e.\ row number $t$ contains the input vector $\mathbf{x}\psupt$.  We also define the column vector $\mathbf{z} = (z\psup{1}, \ldots, z\psup{m})$ of outcomes and recall that $\mathbf{w} = (w_1, \ldots, w_n)$ is a column vector of SLP weights.  The batch cost can now be written as a dot product
\begin{equation}
  \label{eq:lsr-matrix}
  E(\mathbf{w}) = \bigl( \mathbf{z} - \mathbf{X} \mathbf{w} \bigr)^T \bigl( \mathbf{z} - \mathbf{X} \mathbf{w} \bigr) .
\end{equation}
The least-squares solution must satisfy the condition $\nabla E(\mathbf{w}) = \mathbf{0}$, which leads to the so-called normal equations
\begin{equation}
  \label{eq:lsr-normal}
  \mathbf{X}^T \mathbf{X} \mathbf{w} = \mathbf{X}^T \mathbf{z} .
\end{equation}
In the case of the W-H rule, $\mathbf{X}$ is a coincidence matrix between cues and events, with $x_{ti} \in \{0, 1\}$.  A straightforward calculation shows that $\mathbf{X}^T \mathbf{X}$ is a square matrix of co-occurrence counts $f(C_i, C_j)$ between cues, and $\mathbf{X}^T \mathbf{z}$ is a vector of co-occurrence counts $f(C_i, O)$ between the cues and the outcome $O$.  Dividing Eq.~(\ref{eq:lsr-normal}) by $m$, we obtain Eq.~(\ref{eq:danks-probs}) with $\lambda = 1$, i.e.
\[
P(O, C_i) - \sum_{j=1}^n V_j P(C_i, C_j) = 0
\]
which is the same as Eq.~(3) of \citet{Danks:03} with rows multiplied by $P(C_i)$.  Since linear regression is invariant wrt.\ the salience factors $a_i$ (the weights are simply adjusted by reciprocal factors $1 / a_i$ to achieve the same regression values) and scales linearly with $\lambda$, equivalence to the equilibrium conditions \citep[112--114]{Danks:03} also holds for arbitrary values of $a_i$ and $\lambda$.

In the general case where the regression problem has a unique solution, $\mathbf{X}^T \mathbf{X}$ is symmetric and positive definite.  It can therefore be inverted and the least-squares solution is given by
\begin{equation}
  \label{eq:lsr-solution}
  \mathbf{w}^* = (\mathbf{X}^T \mathbf{X})^{-1} \mathbf{X}^T \mathbf{z} .
\end{equation}
Standard statistical software such as R \citep{R:10} can be used to compute $\mathbf{w}^*$ reliably and efficiently.  It is not necessary to carry out the iterative training procedure of the R-W model or the neural network, and there is no need to worry about convergence of the iterative training.

\section{Consequences}
\label{sec:consequence}

\begin{itemize}

\item We have shown that R-W association learning, a linear SLP neural network and linear regression are fully equivalent and should ideally lead to the same least-squares solution.  As long as a researcher is only interested in the result of association learning, not in the iterative process, it is sufficient to calculate the least-squares solution directly from Eq.~(\ref{eq:lsr-solution}).

\item The R-W salience factors $\alpha_i$ have no effect on the learning result -- because linear regression is not sensitive to such a scaling of the input variables -- but only on the learning process: associations for cues with high salience $\alpha_i$ are learnt faster than for other cues.  The parameter $\lambda$ leads to a (trivial) linear scaling of the learning result, but has not effect on the learning process.  Only different learning rates $\beta_1\neq \beta_2$ affect the learning result, because they modify $\mathbf{X}^T \mathbf{X}$ in a complex way.

\item If R-W association learning or SLP training does not approximate the least-squares solution, it can arguably be considered to have failed.  The only research question of interest that requires R-W iteration or application of the delta rule is thus: Under which circumstances and for which parameter settings does the R-W iteration converge or at least approximate the linear regresson? This is particularly relevant for single-event updates (as specified for the R-W model), which are much less robust and lead to larger fluctuations then batch updates.  We plan to work on these issues with the help of simulation experiments.  

\item Having established NDL as linear regression with its well-known drawbacks (e.g.\ a propensity for overfitting the training data, especially if there is a large number $n$ of cues), it will be interesting to contrast it with more state-of-the-art machine learning techniques.  We plan to carry out a mathematical analysis and empirical study of (i) logistic regression, which is more appropriate for dichotomous data than linear least-squares regression, and (ii) regularization techniques, which control overfitting and encourage sparse solutions.

\end{itemize}

%%%%%%%%%%%%%%%%%%%%%%%%%%%%%%%%%%%%%%%%%%%%%%%%%%%%%%%%%%%%%%%%%%%%%%%%
%% -- the biblography (automatically generated with bibtex)
%% \renewcommand{\bibsection}{LITERATUR}  % change heading of references section
\bibliographystyle{natbib-modified}       % there are many other bibliography styles
\bibliography{stefan-literature,stefan-publications}

\end{document}
