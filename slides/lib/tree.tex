%%
%% macros for typesetting parse trees with PGF
%%

\usepackage{calc}

%% configure tree dimensions:
%%   \setlength{\treeHstep}{5mm}, \setlength{\treeVstep}{5mm}
%%   \setTreeHeight{6} -> 6 non-terminal layers above terminal string (default: 8)
%%   \setTreeRelative{3.5} -> slots are specified relative to this position (default: 0)
%%   \setTreeBoxW{Text}, \setTreeBoxH{Text}, \setTreeBoxWH{Text} 
%%      -> set amount of space reserved for tree node (to accommodate Text in current font)
%%
\newlength{\treeHstep}
\setlength{\treeHstep}{10mm}
\newlength{\treeVstep}
\setlength{\treeVstep}{10mm}
\newlength{\treeBoxH}
\newlength{\treeBoxD}
\newlength{\treeBoxW}
%% internal lengths and commands
\newlength{\treeBoxTotalH}
\newlength{\treeBoxBaseDY}
\newcommand{\calculateTreeBox}{%
  \setlength{\treeBoxTotalH}{\treeBoxH + \treeBoxD}%
  \setlength{\treeBoxBaseDY}{\treeBoxD - \treeBoxTotalH / 2}%
}
\newcommand{\setTreeBoxW}[1]{\settowidth{\treeBoxW}{#1}}
\newcommand{\setTreeBoxH}[1]{%
  \settoheight{\treeBoxH}{#1}%
  \settodepth{\treeBoxD}{#1}%
  \calculateTreeBox{}}
\newcommand{\setTreeBoxWH}[1]{\setTreeBoxW{#1}\setTreeBoxH{#1}}
\setTreeBoxW{XX}                % default: W = two wide letters
\setTreeBoxH{Tg}                % default: H = full height + depth
\newcounter{TreeHeight}
\setcounter{TreeHeight}{8}
\newcommand{\setTreeHeight}[1]{\setcounter{TreeHeight}{#1}}
%% internal lengths and commands
\newlength{\treeRelativeX}
\setlength{\treeRelativeX}{0mm}
\newcommand{\setTreeRelative}[1]{\setlength{\treeRelativeX}{\treeHstep * \real{#1}}}

%% these are temporary registers for internal use
\newlength{\treeTemp}
\newlength{\treeTempX}
\newlength{\treeTempY}
\newcounter{TreeTemp}

%% internal: \treeYforLayer{<layer>} -> \treeTempY = y-position
\newcommand{\treeYforLayer}[1]{%
  \setlength{\treeTempY}{\treeVstep * \real{\theTreeHeight} - \treeVstep * \real{#1}}%
}

%% internal: \treeXforSlot{<slot>} -> \treeTempX = x-position
\newcommand{\treeXforSlot}[1]{%
  \setlength{\treeTempX}{\treeRelativeX + \treeHstep * \real{#1}}%
}

%% internal: insert strut for full text line height (to improve text alignment)
\newcommand{\treeStrut}{%
  \settoheight{\treeTemp}{X}\rule{0mm}{\treeTemp}%
}

%% put (non-terminal) tree node in slot <Slot> on layer <Layer>, 
%% labelled with <Text> and assigned the node label <Lab>; 
%% layers count from 0 = root layer at the top of the tree
%%   \treeNode<overlay>{Lab}{Layer}{Slot}{Text}
%%   \treeNodePT<overlay>{Lab}{Slot}{Text} ... on pre-terminal layer
\newcommand<>{\treeNode}[4]{%
  \treeYforLayer{#2}\treeXforSlot{#3}%
  \pgfnoderect{#1}[virtual]{\pgfpoint{\treeTempX}{\treeTempY}}{\pgfpoint{\treeBoxW}{\treeBoxTotalH}}%
  \addtolength{\treeTempY}{\treeBoxBaseDY}%
  \only#5{\pgfputat{\pgfpoint{\treeTempX}{\treeTempY}}{\pgfbox[center,base]{#4}}}%
}
\newcommand<>{\treeNodePT}[3]{%
  \setcounter{TreeTemp}{\theTreeHeight - 1}%
  \treeNode#4{#1}{\theTreeTemp}{#2}{#3}%
}

%% put (terminal) leaf node in slot <Slot> on lexical layer>, 
%% labelled with <Text> and assigned the node label <Lab>
%%   \treeLeaf<overlay>{Lab}{Slot}{Text}
\newcommand<>{\treeLeaf}[3]{%
  \treeNode#4{#1}{\theTreeHeight}{#2}{#3}%
}

%% connect tree nodes with straight (unlabelled) edge
%%   \treeEdge{Lab1}{Lab2}
\newcommand<>{\treeEdge}[2]{%
  \begin{pgfscope}
    \pgfclearstartarrow
    \pgfclearendarrow
    \pgfsetlinewidth{1pt}
    \pgfnodeconnline#3{#1}{#2}
  \end{pgfscope}
}

%% connect tree nodes with curved (unlabelled) edge; leaves upper node
%% Lab1 at Angle relative to vertical, enters lower node Lab2 vertically
%%   \treeEdgeCurve{Lab1}{Angle}{Lab2}
\newcommand<>{\treeEdgeCurve}[3]{%
  \begin{pgfscope}
    \pgfclearstartarrow%
    \pgfclearendarrow%
    \pgfsetlinewidth{1pt}%
    \setcounter{TreeTemp}{#2 - 90}%
    \pgfnodeconncurve#4{#1}{#3}{\theTreeTemp}{90}{\treeVstep}{\treeVstep}
  \end{pgfscope}
}

%% attach note to given node (in right bottom corner)
%%   \treeNote<overlay>[right]{Lab}{Text}
\newcommand<>{\treeNote}[3][left]{%
  \only#4{%
    \pgfputat{\pgfnodeborder{#2}{-60}{-2pt}}{%
      \pgfbox[#1,top]{\setlength{\fboxsep}{1pt}\colorbox{white}{%
          \scriptsize\secondary{\treeStrut{}#3}}}}}%
}

%% represent subtree (root=Root, boundary=Left..Right) as triangle
%%   \treeTriangle<overlay>{Root}{Left}{Right}
\newcommand<>{\treeTriangle}[3]{%
  \begin{pgfscope}
    \only#4{%
      \pgfclearstartarrow%
      \pgfclearendarrow%
      \pgfsetlinewidth{1pt}%
      \pgfmoveto{\pgfnodeborder{#1}{-90}{2pt}}%
      \pgflineto{\pgfnodeborder{#2}{135}{2pt}}%
      \pgflineto{\pgfnodeborder{#3}{45}{2pt}}%
      \pgfclosestroke%
    }
  \end{pgfscope}
}



%%% Local Variables: 
%%% mode: latex
%%% TeX-master: ""
%%% End: 
